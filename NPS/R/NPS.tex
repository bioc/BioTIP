\documentclass[]{article}
\usepackage{lmodern}
\usepackage{amssymb,amsmath}
\usepackage{ifxetex,ifluatex}
\usepackage{fixltx2e} % provides \textsubscript
\ifnum 0\ifxetex 1\fi\ifluatex 1\fi=0 % if pdftex
  \usepackage[T1]{fontenc}
  \usepackage[utf8]{inputenc}
\else % if luatex or xelatex
  \ifxetex
    \usepackage{mathspec}
  \else
    \usepackage{fontspec}
  \fi
  \defaultfontfeatures{Ligatures=TeX,Scale=MatchLowercase}
\fi
% use upquote if available, for straight quotes in verbatim environments
\IfFileExists{upquote.sty}{\usepackage{upquote}}{}
% use microtype if available
\IfFileExists{microtype.sty}{%
\usepackage{microtype}
\UseMicrotypeSet[protrusion]{basicmath} % disable protrusion for tt fonts
}{}
\usepackage[margin=1in]{geometry}
\usepackage{hyperref}
\hypersetup{unicode=true,
            pdftitle={NPS.R},
            pdfauthor={benjk},
            pdfborder={0 0 0},
            breaklinks=true}
\urlstyle{same}  % don't use monospace font for urls
\usepackage{color}
\usepackage{fancyvrb}
\newcommand{\VerbBar}{|}
\newcommand{\VERB}{\Verb[commandchars=\\\{\}]}
\DefineVerbatimEnvironment{Highlighting}{Verbatim}{commandchars=\\\{\}}
% Add ',fontsize=\small' for more characters per line
\usepackage{framed}
\definecolor{shadecolor}{RGB}{248,248,248}
\newenvironment{Shaded}{\begin{snugshade}}{\end{snugshade}}
\newcommand{\AlertTok}[1]{\textcolor[rgb]{0.94,0.16,0.16}{#1}}
\newcommand{\AnnotationTok}[1]{\textcolor[rgb]{0.56,0.35,0.01}{\textbf{\textit{#1}}}}
\newcommand{\AttributeTok}[1]{\textcolor[rgb]{0.77,0.63,0.00}{#1}}
\newcommand{\BaseNTok}[1]{\textcolor[rgb]{0.00,0.00,0.81}{#1}}
\newcommand{\BuiltInTok}[1]{#1}
\newcommand{\CharTok}[1]{\textcolor[rgb]{0.31,0.60,0.02}{#1}}
\newcommand{\CommentTok}[1]{\textcolor[rgb]{0.56,0.35,0.01}{\textit{#1}}}
\newcommand{\CommentVarTok}[1]{\textcolor[rgb]{0.56,0.35,0.01}{\textbf{\textit{#1}}}}
\newcommand{\ConstantTok}[1]{\textcolor[rgb]{0.00,0.00,0.00}{#1}}
\newcommand{\ControlFlowTok}[1]{\textcolor[rgb]{0.13,0.29,0.53}{\textbf{#1}}}
\newcommand{\DataTypeTok}[1]{\textcolor[rgb]{0.13,0.29,0.53}{#1}}
\newcommand{\DecValTok}[1]{\textcolor[rgb]{0.00,0.00,0.81}{#1}}
\newcommand{\DocumentationTok}[1]{\textcolor[rgb]{0.56,0.35,0.01}{\textbf{\textit{#1}}}}
\newcommand{\ErrorTok}[1]{\textcolor[rgb]{0.64,0.00,0.00}{\textbf{#1}}}
\newcommand{\ExtensionTok}[1]{#1}
\newcommand{\FloatTok}[1]{\textcolor[rgb]{0.00,0.00,0.81}{#1}}
\newcommand{\FunctionTok}[1]{\textcolor[rgb]{0.00,0.00,0.00}{#1}}
\newcommand{\ImportTok}[1]{#1}
\newcommand{\InformationTok}[1]{\textcolor[rgb]{0.56,0.35,0.01}{\textbf{\textit{#1}}}}
\newcommand{\KeywordTok}[1]{\textcolor[rgb]{0.13,0.29,0.53}{\textbf{#1}}}
\newcommand{\NormalTok}[1]{#1}
\newcommand{\OperatorTok}[1]{\textcolor[rgb]{0.81,0.36,0.00}{\textbf{#1}}}
\newcommand{\OtherTok}[1]{\textcolor[rgb]{0.56,0.35,0.01}{#1}}
\newcommand{\PreprocessorTok}[1]{\textcolor[rgb]{0.56,0.35,0.01}{\textit{#1}}}
\newcommand{\RegionMarkerTok}[1]{#1}
\newcommand{\SpecialCharTok}[1]{\textcolor[rgb]{0.00,0.00,0.00}{#1}}
\newcommand{\SpecialStringTok}[1]{\textcolor[rgb]{0.31,0.60,0.02}{#1}}
\newcommand{\StringTok}[1]{\textcolor[rgb]{0.31,0.60,0.02}{#1}}
\newcommand{\VariableTok}[1]{\textcolor[rgb]{0.00,0.00,0.00}{#1}}
\newcommand{\VerbatimStringTok}[1]{\textcolor[rgb]{0.31,0.60,0.02}{#1}}
\newcommand{\WarningTok}[1]{\textcolor[rgb]{0.56,0.35,0.01}{\textbf{\textit{#1}}}}
\usepackage{graphicx,grffile}
\makeatletter
\def\maxwidth{\ifdim\Gin@nat@width>\linewidth\linewidth\else\Gin@nat@width\fi}
\def\maxheight{\ifdim\Gin@nat@height>\textheight\textheight\else\Gin@nat@height\fi}
\makeatother
% Scale images if necessary, so that they will not overflow the page
% margins by default, and it is still possible to overwrite the defaults
% using explicit options in \includegraphics[width, height, ...]{}
\setkeys{Gin}{width=\maxwidth,height=\maxheight,keepaspectratio}
\IfFileExists{parskip.sty}{%
\usepackage{parskip}
}{% else
\setlength{\parindent}{0pt}
\setlength{\parskip}{6pt plus 2pt minus 1pt}
}
\setlength{\emergencystretch}{3em}  % prevent overfull lines
\providecommand{\tightlist}{%
  \setlength{\itemsep}{0pt}\setlength{\parskip}{0pt}}
\setcounter{secnumdepth}{0}
% Redefines (sub)paragraphs to behave more like sections
\ifx\paragraph\undefined\else
\let\oldparagraph\paragraph
\renewcommand{\paragraph}[1]{\oldparagraph{#1}\mbox{}}
\fi
\ifx\subparagraph\undefined\else
\let\oldsubparagraph\subparagraph
\renewcommand{\subparagraph}[1]{\oldsubparagraph{#1}\mbox{}}
\fi

%%% Use protect on footnotes to avoid problems with footnotes in titles
\let\rmarkdownfootnote\footnote%
\def\footnote{\protect\rmarkdownfootnote}

%%% Change title format to be more compact
\usepackage{titling}

% Create subtitle command for use in maketitle
\providecommand{\subtitle}[1]{
  \posttitle{
    \begin{center}\large#1\end{center}
    }
}

\setlength{\droptitle}{-2em}

  \title{NPS.R}
    \pretitle{\vspace{\droptitle}\centering\huge}
  \posttitle{\par}
    \author{benjk}
    \preauthor{\centering\large\emph}
  \postauthor{\par}
      \predate{\centering\large\emph}
  \postdate{\par}
    \date{2019-06-05}


\begin{document}
\maketitle

Classifying Transcriptomes into Biotypes

@aliases getBiotypes

@description The purpose of the getBiotypes function is to class both
coding and noncoding transcripts into biotypes using the most recent
GENCODE annotations. This tool can also be used to define potential
lncRNAs, given an available genome transcriptome assembly (a gtf file)
or any genomic loci of interest.

@param full\_gr A GRanges object of coding and noncoding transctipts.
Unique identifications for each column must be assigned.More details can
be found in the GRanges package. @param gencode\_gr This GRanges object
contains a GENCODE reference annotation.It must have a column of
biotypes. @param intron\_gr A GRanges object containing the coordinates
of introns.For details see GRanges package. @param minoverlap Detects
minimum overlap between two IRanges objects. Details Overlap arguments
are included in the IRanges package.

@details For details of findOverlaps, type.partialOverlap,
type.50Overlap type.toPlot, queryhits, and subjecthits see
\href{https://www.bioconductor.org/packages/release/bioc/html/GenomicRanges.html}{GenomicRanges}
\href{https://www.bioconductor.org/packages/release/bioc/html/IRanges.html}{IRanges},
and \href{http://bioconductor.org/install/index.html}{BiocManager}.

@return Returns the classified transcriptome biotypes.

@source
\href{https://www.gencodegenes.org/human/release_25lift37.html}{Refrence
GRCh37 genome} for details on gtf format visit
\href{https://useast.ensembl.org/info/website/upload/gff.html}{ensemble}
@import GenomicRanges

@references

Wang, Z. Z., J. M. Cunningham and X. H. Yang (2018).'CisPi: a
transcriptomic score for disclosing cis-acting disease-associated
lincRNAs.' Bioinformatics34(17): 664-670'

@examples \#Input datasets locally library(``GenomicRanges'' ,
``IRanges'')

data(``gencode'') data(``intron'') data(``ILEF'') gencode\_gr =
GRanges(gencode) ILEF\_gr = GRanges(ILEF) cod\_gr = GRanges(cod)

getBiotypes(ILEF\_gr,gencode\_gr)

\dontrun{getBiotypes(intron_gr)}

@note Replace the path\_file when loading data locally to the data
directory.

@import GenomicRanges IRanges GenomeInfoDb stats @importFrom stats
aggregate @export

\begin{Shaded}
\begin{Highlighting}[]
\NormalTok{getBiotypes <-}\StringTok{ }\ControlFlowTok{function}\NormalTok{(full_gr, gencode_gr, }\DataTypeTok{intron_gr =} \OtherTok{NULL}\NormalTok{, }\DataTypeTok{minoverlap =}\NormalTok{ 1L) \{}
    \ControlFlowTok{if}\NormalTok{ (}\KeywordTok{class}\NormalTok{(full_gr) }\OperatorTok{!=}\StringTok{ "GRanges"}\NormalTok{)}
        \KeywordTok{stop}\NormalTok{(}\StringTok{"please give full_gr as a }\CharTok{\textbackslash{}"}\StringTok{GRanges}\CharTok{\textbackslash{}"}\StringTok{ object"}\NormalTok{)}
    \ControlFlowTok{if}\NormalTok{ (}\KeywordTok{class}\NormalTok{(gencode_gr) }\OperatorTok{!=}\StringTok{ "GRanges"}\NormalTok{)}
        \KeywordTok{stop}\NormalTok{(}\StringTok{"pealse give gencode_gr as a }\CharTok{\textbackslash{}"}\StringTok{GRanges}\CharTok{\textbackslash{}"}\StringTok{ object"}\NormalTok{)}
    \ControlFlowTok{if}\NormalTok{ (}\KeywordTok{class}\NormalTok{(intron_gr) }\OperatorTok{!=}\StringTok{ "GRanges"} \OperatorTok{&}\StringTok{ }\OperatorTok{!}\KeywordTok{is.null}\NormalTok{(intron_gr))}
        \KeywordTok{stop}\NormalTok{(}\StringTok{"please give intron_gr as a }\CharTok{\textbackslash{}"}\StringTok{GRanges}\CharTok{\textbackslash{}"}\StringTok{ object"}\NormalTok{)}
\NormalTok{    hits =}\StringTok{ }\KeywordTok{findOverlaps}\NormalTok{(full_gr, gencode_gr, }\DataTypeTok{type =} \StringTok{"within"}\NormalTok{, }\DataTypeTok{minoverlap =}\NormalTok{ minoverlap)}
\NormalTok{    full =}\StringTok{ }\KeywordTok{as.data.frame}\NormalTok{(full_gr)}
\NormalTok{    full}\OperatorTok{$}\NormalTok{type.fullOverlap =}\StringTok{ "de novo"}
\NormalTok{    idx =}\StringTok{ }\KeywordTok{as.data.frame}\NormalTok{(}\KeywordTok{mcols}\NormalTok{(full_gr[}\KeywordTok{queryHits}\NormalTok{(hits)]))}
    \ControlFlowTok{if}\NormalTok{ (}\KeywordTok{nrow}\NormalTok{(idx) }\OperatorTok{!=}\StringTok{ }\DecValTok{0}\NormalTok{) \{}
\NormalTok{        idx}\OperatorTok{$}\NormalTok{biotype =}\StringTok{ }\KeywordTok{as.data.frame}\NormalTok{(}\KeywordTok{mcols}\NormalTok{(gencode_gr[}\KeywordTok{subjectHits}\NormalTok{(hits)]))[, }\DecValTok{1}\NormalTok{]}
\NormalTok{        idx_collapse =}\StringTok{ }\KeywordTok{aggregate}\NormalTok{(}\KeywordTok{as.list}\NormalTok{(idx[}\StringTok{"biotype"}\NormalTok{]), idx[}\StringTok{"Row.names"}\NormalTok{], }\DataTypeTok{FUN =} \ControlFlowTok{function}\NormalTok{(X) }\KeywordTok{paste}\NormalTok{(}\KeywordTok{unique}\NormalTok{(X),}
            \DataTypeTok{collapse =} \StringTok{", "}\NormalTok{))}
\NormalTok{        idx_full =}\StringTok{ }\KeywordTok{match}\NormalTok{(idx_collapse}\OperatorTok{$}\NormalTok{Row.names, full}\OperatorTok{$}\NormalTok{Row.names)}
\NormalTok{        full[idx_full, ]}\OperatorTok{$}\NormalTok{type.fullOverlap =}\StringTok{ }\NormalTok{idx_collapse}\OperatorTok{$}\NormalTok{biotype\}}
\NormalTok{    hits =}\StringTok{ }\KeywordTok{findOverlaps}\NormalTok{(full_gr, gencode_gr, }\DataTypeTok{minoverlap =}\NormalTok{ minoverlap)}
\NormalTok{    overlaps <-}\StringTok{ }\KeywordTok{pintersect}\NormalTok{(full_gr[}\KeywordTok{queryHits}\NormalTok{(hits)], gencode_gr[}\KeywordTok{subjectHits}\NormalTok{(hits)])}
\NormalTok{    percentOverlap <-}\StringTok{ }\KeywordTok{width}\NormalTok{(overlaps)}\OperatorTok{/}\KeywordTok{width}\NormalTok{(gencode_gr[}\KeywordTok{subjectHits}\NormalTok{(hits)])}
\NormalTok{    idx =}\StringTok{ }\KeywordTok{as.data.frame}\NormalTok{(}\KeywordTok{mcols}\NormalTok{(full_gr[}\KeywordTok{queryHits}\NormalTok{(hits)]))}
\NormalTok{    idx}\OperatorTok{$}\NormalTok{biotype =}\StringTok{ }\KeywordTok{as.data.frame}\NormalTok{(}\KeywordTok{mcols}\NormalTok{(gencode_gr[}\KeywordTok{subjectHits}\NormalTok{(hits)]))}
\NormalTok{    idx_collapse =}\StringTok{ }\KeywordTok{aggregate}\NormalTok{(}\KeywordTok{as.list}\NormalTok{(idx[}\StringTok{"biotype"}\NormalTok{]), idx[}\StringTok{"Row.names"}\NormalTok{], }\DataTypeTok{FUN =} \ControlFlowTok{function}\NormalTok{(X) }\KeywordTok{paste}\NormalTok{(}\KeywordTok{unique}\NormalTok{(X),}
        \DataTypeTok{collapse =} \StringTok{", "}\NormalTok{))}
\NormalTok{    full}\OperatorTok{$}\NormalTok{type.partialOverlap =}\StringTok{ "de novo"}
\NormalTok{    idx_partial =}\StringTok{ }\KeywordTok{match}\NormalTok{(idx_collapse}\OperatorTok{$}\NormalTok{Row.names, full}\OperatorTok{$}\NormalTok{Row.names)}
\NormalTok{    full[idx_partial, ]}\OperatorTok{$}\NormalTok{type.partialOverlap =}\StringTok{ }\NormalTok{idx_collapse}\OperatorTok{$}\NormalTok{biotype}
\NormalTok{    idx}\OperatorTok{$}\NormalTok{percentOverlap =}\StringTok{ }\NormalTok{percentOverlap}
\NormalTok{    idx_}\DecValTok{50}\NormalTok{ =}\StringTok{ }\KeywordTok{subset}\NormalTok{(idx, percentOverlap }\OperatorTok{>=}\StringTok{ }\FloatTok{0.5}\NormalTok{)}
\NormalTok{    idx_50collapse =}\StringTok{ }\KeywordTok{aggregate}\NormalTok{(}\KeywordTok{as.list}\NormalTok{(idx_}\DecValTok{50}\NormalTok{[}\StringTok{"biotype"}\NormalTok{]), idx_}\DecValTok{50}\NormalTok{[}\StringTok{"Row.names"}\NormalTok{], }\DataTypeTok{FUN =} \ControlFlowTok{function}\NormalTok{(X) }\KeywordTok{paste}\NormalTok{(}\KeywordTok{unique}\NormalTok{(X),}
        \DataTypeTok{collapse =} \StringTok{", "}\NormalTok{))}
\NormalTok{    full}\OperatorTok{$}\NormalTok{type}\FloatTok{.50}\NormalTok{Overlap =}\StringTok{ "de novo"}
\NormalTok{    idx_}\DecValTok{50}\NormalTok{ =}\StringTok{ }\KeywordTok{match}\NormalTok{(idx_50collapse}\OperatorTok{$}\NormalTok{Row.names, full}\OperatorTok{$}\NormalTok{Row.names)}
\NormalTok{    full[idx_}\DecValTok{50}\NormalTok{, ]}\OperatorTok{$}\NormalTok{type}\FloatTok{.50}\NormalTok{Overlap =}\StringTok{ }\NormalTok{idx_50collapse}\OperatorTok{$}\NormalTok{biotype}
    \ControlFlowTok{if}\NormalTok{ (}\OperatorTok{!}\KeywordTok{is.null}\NormalTok{(intron_gr)) \{}
\NormalTok{        hits =}\StringTok{ }\KeywordTok{findOverlaps}\NormalTok{(full_gr, intron_gr)}
\NormalTok{        idx =}\StringTok{ }\KeywordTok{unique}\NormalTok{(}\KeywordTok{as.data.frame}\NormalTok{(}\KeywordTok{mcols}\NormalTok{(full_gr[}\KeywordTok{queryHits}\NormalTok{(hits)])))}
\NormalTok{        full}\OperatorTok{$}\NormalTok{hasIntron =}\StringTok{ "no"}
\NormalTok{        idx_intron =}\StringTok{ }\KeywordTok{match}\NormalTok{(idx}\OperatorTok{$}\NormalTok{Row.names, full}\OperatorTok{$}\NormalTok{Row.names)}
        \ControlFlowTok{if}\NormalTok{ (}\KeywordTok{length}\NormalTok{(idx_intron) }\OperatorTok{!=}\StringTok{ }\DecValTok{0}\NormalTok{)}
\NormalTok{            full[idx_intron, ]}\OperatorTok{$}\NormalTok{hasIntron =}\StringTok{ "yes"}
\NormalTok{    \} }\ControlFlowTok{else}\NormalTok{ (full}\OperatorTok{$}\DataTypeTok{hasIntron =} \OtherTok{NA}\NormalTok{)}
\NormalTok{    full}\OperatorTok{$}\NormalTok{type.toPlot =}\StringTok{ }\KeywordTok{ifelse}\NormalTok{(full}\OperatorTok{$}\NormalTok{hasIntron }\OperatorTok{==}\StringTok{ "yes"} \OperatorTok{&}\StringTok{ }\NormalTok{full}\OperatorTok{$}\NormalTok{type}\FloatTok{.50}\NormalTok{Overlap }\OperatorTok{==}\StringTok{ "protein_coding"}\NormalTok{, }\StringTok{"protein_coding_intron"}\NormalTok{,}
\NormalTok{        full}\OperatorTok{$}\NormalTok{type}\FloatTok{.50}\NormalTok{Overlap)}
\NormalTok{    full}\OperatorTok{$}\NormalTok{type.toPlot =}\StringTok{ }\KeywordTok{sapply}\NormalTok{(full}\OperatorTok{$}\NormalTok{type.toPlot, }\ControlFlowTok{function}\NormalTok{(x) }\KeywordTok{ifelse}\NormalTok{(}\KeywordTok{grepl}\NormalTok{(}\StringTok{"protein_coding"}\NormalTok{, x) }\OperatorTok{&}\StringTok{ }\KeywordTok{grepl}\NormalTok{(}\StringTok{"antisense"}\NormalTok{,}
\NormalTok{        x), }\StringTok{"protein_coding_antisense"}\NormalTok{, x))}
\NormalTok{    full}\OperatorTok{$}\NormalTok{type.toPlot =}\StringTok{ }\KeywordTok{sapply}\NormalTok{(full}\OperatorTok{$}\NormalTok{type.toPlot, }\ControlFlowTok{function}\NormalTok{(x) }\KeywordTok{ifelse}\NormalTok{(}\KeywordTok{grepl}\NormalTok{(}\StringTok{"protein_coding,"}\NormalTok{, x), }\StringTok{"protein_coding_mixed"}\NormalTok{,}
\NormalTok{        x))}
\NormalTok{    full}\OperatorTok{$}\NormalTok{type.toPlot =}\StringTok{ }\KeywordTok{sapply}\NormalTok{(full}\OperatorTok{$}\NormalTok{type.toPlot, }\ControlFlowTok{function}\NormalTok{(x) }\KeywordTok{ifelse}\NormalTok{(}\KeywordTok{grepl}\NormalTok{(}\StringTok{", protein_coding"}\NormalTok{, x), }\StringTok{"protein_coding_mixed"}\NormalTok{,}
\NormalTok{        x))}
\NormalTok{    full}\OperatorTok{$}\NormalTok{type.toPlot =}\StringTok{ }\KeywordTok{sapply}\NormalTok{(full}\OperatorTok{$}\NormalTok{type.toPlot, }\ControlFlowTok{function}\NormalTok{(x) }\KeywordTok{ifelse}\NormalTok{(}\KeywordTok{grepl}\NormalTok{(}\StringTok{"lincRNA"}\NormalTok{, x), }\StringTok{"lincRNA"}\NormalTok{,}
\NormalTok{        x))}
\NormalTok{    full}\OperatorTok{$}\NormalTok{type.toPlot =}\StringTok{ }\KeywordTok{sapply}\NormalTok{(full}\OperatorTok{$}\NormalTok{type.toPlot, }\ControlFlowTok{function}\NormalTok{(x) }\KeywordTok{ifelse}\NormalTok{(}\KeywordTok{grepl}\NormalTok{(}\StringTok{"antisense,"}\NormalTok{, x), }\StringTok{"antisense"}\NormalTok{,}
\NormalTok{        x))}
\NormalTok{    full}\OperatorTok{$}\NormalTok{type.toPlot =}\StringTok{ }\KeywordTok{sapply}\NormalTok{(full}\OperatorTok{$}\NormalTok{type.toPlot, }\ControlFlowTok{function}\NormalTok{(x) }\KeywordTok{ifelse}\NormalTok{(}\KeywordTok{grepl}\NormalTok{(}\StringTok{", antisense"}\NormalTok{, x), }\StringTok{"antisense"}\NormalTok{,}
\NormalTok{        x))}
\NormalTok{    label =}\StringTok{ }\KeywordTok{c}\NormalTok{(}\StringTok{"protein_coding"}\NormalTok{, }\StringTok{"protein_coding_mixed"}\NormalTok{, }\StringTok{"lincRNA"}\NormalTok{, }\StringTok{"antisense"}\NormalTok{, }\StringTok{"pseudogene, processed_pseudogene"}\NormalTok{,}
        \StringTok{"pseudogene, unprocessed_pseudogene"}\NormalTok{, }\StringTok{"de novo"}\NormalTok{, }\StringTok{"protein_coding_antisense"}\NormalTok{, }\StringTok{"protein_coding_intron"}\NormalTok{,}
        \StringTok{"miRNA"}\NormalTok{)}
\NormalTok{    full}\OperatorTok{$}\NormalTok{type.toPlot =}\StringTok{ }\KeywordTok{sapply}\NormalTok{(full}\OperatorTok{$}\NormalTok{type.toPlot, }\ControlFlowTok{function}\NormalTok{(x) }\KeywordTok{ifelse}\NormalTok{(}\OperatorTok{!}\NormalTok{x }\OperatorTok\StringTok{ }\NormalTok{label, }\StringTok{"other_noncoding"}\NormalTok{,}
\NormalTok{        x))}
    \KeywordTok{return}\NormalTok{(full)}
\NormalTok{\}}
\end{Highlighting}
\end{Shaded}

Finding Overlaps in Coding Regions

@description A getReadthrough functions that will The getReadthrough
function is used to find long transcripts that covers more than two
coding regions of a genome. The variables are as follows.

@param gr A GRanges object that shows the start and end loci on genome.
@param cod\_gr A GRanges object that contains coding regions. For
details please visit /R/data.R.

@details For details of findOverlaps, type.partialOverlap,
type.50Overlap type.toPlot, queryhits, readthrough and subjecthits see,
\href{https://www.bioconductor.org/packages/release/bioc/html/GenomicRanges.html}{GenomicRanges},
\href{https://www.bioconductor.org/packages/release/bioc/html/IRanges.html}{IRanges},
and \href{http://bioconductor.org/install/index.html}{BiocManager}.

@return Returns the classified transcriptome biotypes.

@source
\href{https://www.gencodegenes.org/human/release_25lift37.html}{Refrence
GRCh37 genome} for details on gtf format visit
\href{https://useast.ensembl.org/info/website/upload/gff.html}{ensemble}

@import GenomeInfoDb GenomicRanges @importFrom stats aggregate

@references Wang, Z. Z., J. M. Cunningham and X. H. Yang (2018).'CisPi:
a transcriptomic score for disclosing cis-acting disease-associated
lincRNAs.' Bioinformatics34(17): 664-670'

@examples \#First Load datasets library(``GenomicRanges'' , ``IRanges'')

data(``gencode'') data(``ILEF'') data(``cod'') gencode\_gr =
GRanges(gencode) ILEF\_gr = GRanges(ILEF) cod\_gr = GRanges(cod)

getReadthrough(ILEF\_gr,cod\_gr)

\dontrun{getReadthrough(cod_gr)}

@note Replace the path\_file when loading data locally to the data
directory.

@import GenomicRanges IRanges GenomeInfoDb @importFrom stats aggregate
@export

\begin{Shaded}
\begin{Highlighting}[]
\NormalTok{getReadthrough =}\StringTok{ }\ControlFlowTok{function}\NormalTok{(gr,cod_gr)\{}
\NormalTok{  full_table =}\StringTok{ }\KeywordTok{data.frame}\NormalTok{(gr)}
\NormalTok{  overlapcount =}\StringTok{ }\KeywordTok{countOverlaps}\NormalTok{(gr,cod_gr)}
\NormalTok{  completeoverlap =}\StringTok{ }\KeywordTok{unique}\NormalTok{(}\KeywordTok{subjectHits}\NormalTok{(}\KeywordTok{findOverlaps}\NormalTok{(cod_gr,}\KeywordTok{GRanges}\NormalTok{(full_table}\OperatorTok{$}\NormalTok{ID),}\DataTypeTok{type =} \StringTok{'within'}\NormalTok{)))}
  \ControlFlowTok{if}\NormalTok{(}\KeywordTok{length}\NormalTok{(completeoverlap) }\OperatorTok{==}\StringTok{ }\DecValTok{0}\NormalTok{)\{}
\NormalTok{    full_table}\OperatorTok{$}\NormalTok{readthrough =}\StringTok{ }\KeywordTok{ifelse}\NormalTok{(overlapcount}\OperatorTok{>}\DecValTok{2}\NormalTok{,}\DecValTok{1}\NormalTok{,}\DecValTok{0}\NormalTok{)}
\NormalTok{  \}}\ControlFlowTok{else}\NormalTok{\{}
\NormalTok{    full_table}\OperatorTok{$}\NormalTok{readthrough =}\StringTok{ }\KeywordTok{ifelse}\NormalTok{(overlapcount}\OperatorTok{>}\DecValTok{2} \OperatorTok{&}\StringTok{ }\KeywordTok{row.names}\NormalTok{(completeoverlap) }\OperatorTok\StringTok{ }\NormalTok{completeoverlap,}\DecValTok{1}\NormalTok{,}\DecValTok{0}\NormalTok{)}
\NormalTok{  \}}
\NormalTok{  gr =}\StringTok{ }\KeywordTok{GRanges}\NormalTok{(}\KeywordTok{subset}\NormalTok{(full_table,readthrough }\OperatorTok{==}\DecValTok{1}\NormalTok{))}
\NormalTok{  idx =}\StringTok{ }\KeywordTok{subset}\NormalTok{(full_table,readthrough}\OperatorTok{==}\DecValTok{1}\NormalTok{)}\OperatorTok{$}\NormalTok{ID}
\NormalTok{  overlaps =}\StringTok{ }\KeywordTok{as.data.frame}\NormalTok{(}\KeywordTok{findOverlaps}\NormalTok{(gr,cod_gr))}
\NormalTok{  splitoverlaps =}\StringTok{ }\KeywordTok{split}\NormalTok{(overlaps,}\DataTypeTok{f=}\NormalTok{overlaps}\OperatorTok{$}\NormalTok{queryHits)}
  \KeywordTok{table}\NormalTok{(}\KeywordTok{sapply}\NormalTok{(splitoverlaps,nrow)}\OperatorTok{>}\DecValTok{1}\NormalTok{)}
\NormalTok{  cod_grL =}\StringTok{ }\KeywordTok{sapply}\NormalTok{(splitoverlaps,}\ControlFlowTok{function}\NormalTok{(x) cod_gr[x}\OperatorTok{$}\NormalTok{subjectHits])}
\NormalTok{  overlapL =}\StringTok{ }\KeywordTok{sapply}\NormalTok{(cod_grL,}\ControlFlowTok{function}\NormalTok{(x) }\KeywordTok{findOverlaps}\NormalTok{(x))}
\NormalTok{  notoverlap =}\StringTok{ }\KeywordTok{sapply}\NormalTok{(overlapL,}\ControlFlowTok{function}\NormalTok{(x) }\KeywordTok{identical}\NormalTok{(}\KeywordTok{queryHits}\NormalTok{(x),}\KeywordTok{subjectHits}\NormalTok{(x)))}
\NormalTok{  tmp =}\StringTok{ }\KeywordTok{rep}\NormalTok{(}\OtherTok{TRUE}\NormalTok{,}\KeywordTok{nrow}\NormalTok{(full_table))}
\NormalTok{  tmp[full_table}\OperatorTok{$}\NormalTok{readthrough}\OperatorTok{==}\DecValTok{1}\NormalTok{] =}\StringTok{ }\NormalTok{notoverlap}
\NormalTok{  full_table}\OperatorTok{$}\NormalTok{readthrough =}\StringTok{ }\KeywordTok{ifelse}\NormalTok{(full_table}\OperatorTok{$}\NormalTok{readthrough}\OperatorTok{==}\DecValTok{1} \OperatorTok{&}\StringTok{ }\OperatorTok{!}\NormalTok{tmp,}\DecValTok{1}\NormalTok{,}\DecValTok{0}\NormalTok{)}
  \KeywordTok{return}\NormalTok{(full_table)}
\NormalTok{\}}
\end{Highlighting}
\end{Shaded}

The length of a string (in characters).

@param df a count matrix with unique loci row names:ID X column
names:samples. @param samplesL a list of characters with stages as
names. @param cutoff numeric value, if \textless{} 1 automaticlly goes
to select top x\% transcripts if \textgreater{} 1 goes to select by
x-fold more than the selected method (which is either the reference,
other stages or pervious stage) default 0.01. @param method select from
`reference',`other', or `previous'. for `reference', the reference has
to be the first for `previous', make sure sampleL is in the right order
from benign to malign default uses `other'

@return numeric vector giving number of characters in each element of
the character vector. Missing strings have missing length. @export
@examples GSE6136 =
read.table(``C:/Users/benjk/Desktop/GSE6136\_matrix.txt'', header =
TRUE, comment = `!') row.names(GSE6136) =
GSE6136\(ID_REF GSE6136 = GSE6136[,-1] cli = read.delim("C:/Users/benjk/Desktop/GSE6136_cli.txt", head = F) cli = t(cli) colnames(cli) = str_split_fixed(cli[1,],'_',2)[,2] cli = cli[-1,] cli = data.frame(cli) cli[,'cell-type:ch1'] = str_split_fixed(cli\)characteristics\_ch1.1,`:',2){[},2{]}
cli{[},`Ig clonality:ch1'{]} =
str\_split\_fixed(cli\(characteristics_ch1.3,': ',2)[,2] colnames(cli)[colnames(cli) == 'cell-type:ch1'] = 'group' cli\)Row.names
= cli{[},1{]}

dat \textless- GSE6136 df \textless- log2(dat+1) tmp \textless-
names(table(cli\(group)) sampleL <- split(cli[,1],f = cli\)group) test
\textless- sd\_selection(df,sampleL,0.01) View(test)

@author Zhezhen Wang and Biniam Feleke

\begin{Shaded}
\begin{Highlighting}[]
\NormalTok{sd_selection <-}\StringTok{ }\ControlFlowTok{function}\NormalTok{(df,samplesL,}\DataTypeTok{cutoff =} \FloatTok{0.01}\NormalTok{, }\DataTypeTok{method =} \StringTok{'other'}\NormalTok{)\{}
  \ControlFlowTok{if}\NormalTok{(}\KeywordTok{is.null}\NormalTok{(}\KeywordTok{names}\NormalTok{(samplesL))) }\KeywordTok{stop}\NormalTok{(}\StringTok{'please provide name to samplesL'}\NormalTok{)}
\NormalTok{  tmp =}\StringTok{ }\KeywordTok{names}\NormalTok{(samplesL)}
\NormalTok{  samplesL =}\StringTok{ }\KeywordTok{lapply}\NormalTok{(samplesL,as.character)}
\NormalTok{  test2 =}\StringTok{ }\KeywordTok{sapply}\NormalTok{(tmp, }\ControlFlowTok{function}\NormalTok{(x) }\KeywordTok{apply}\NormalTok{(df[,samplesL[[x]]],}\DecValTok{1}\NormalTok{,sd,}\DataTypeTok{na.rm =}\NormalTok{ T))}

  \ControlFlowTok{if}\NormalTok{(method }\OperatorTok{==}\StringTok{ 'reference'}\NormalTok{)\{}
\NormalTok{    ref =}\StringTok{ }\KeywordTok{as.character}\NormalTok{(sampleL[[}\DecValTok{1}\NormalTok{]])}
\NormalTok{    sdref =}\StringTok{ }\KeywordTok{apply}\NormalTok{(df[,ref],}\DecValTok{1}\NormalTok{,sd,}\DataTypeTok{na.rm =}\NormalTok{ T)}
\NormalTok{    sds =}\StringTok{ }\KeywordTok{lapply}\NormalTok{(tmp,}\ControlFlowTok{function}\NormalTok{(x) test2[,x]}\OperatorTok{/}\NormalTok{sdref)}
    \KeywordTok{names}\NormalTok{(sds) =}\StringTok{ }\NormalTok{tmp}

\NormalTok{  \}}\ControlFlowTok{else} \ControlFlowTok{if}\NormalTok{(method }\OperatorTok{==}\StringTok{ 'other'}\NormalTok{)\{}
\NormalTok{    othersample =}\StringTok{ }\KeywordTok{lapply}\NormalTok{(}\DecValTok{1}\OperatorTok{:}\StringTok{ }\KeywordTok{length}\NormalTok{(samplesL), }\ControlFlowTok{function}\NormalTok{(x) }\KeywordTok{do.call}\NormalTok{(c,samplesL[}\OperatorTok{-}\NormalTok{x]))}
    \KeywordTok{names}\NormalTok{(othersample) =}\StringTok{ }\NormalTok{tmp}
\NormalTok{    sdother =}\StringTok{ }\KeywordTok{sapply}\NormalTok{(tmp, }\ControlFlowTok{function}\NormalTok{(x) }\KeywordTok{apply}\NormalTok{(df[,}\KeywordTok{as.character}\NormalTok{(othersample[[x]])],}\DecValTok{1}\NormalTok{,sd,}\DataTypeTok{na.rm =}\NormalTok{ T))}

\NormalTok{    sds =}\StringTok{ }\KeywordTok{lapply}\NormalTok{(tmp,}\ControlFlowTok{function}\NormalTok{(x) test2[,x]}\OperatorTok{/}\NormalTok{sdother[,x])}
    \KeywordTok{names}\NormalTok{(sds) =}\StringTok{ }\NormalTok{tmp}

\NormalTok{  \}}\ControlFlowTok{else} \ControlFlowTok{if}\NormalTok{(method }\OperatorTok{==}\StringTok{ 'previous'}\NormalTok{)\{}
    \KeywordTok{warning}\NormalTok{(}\StringTok{'Using method "previous", make sure sampleL is in the right order'}\NormalTok{)}
\NormalTok{    sds =}\StringTok{ }\KeywordTok{lapply}\NormalTok{(}\DecValTok{2}\OperatorTok{:}\KeywordTok{ncol}\NormalTok{(test2),}\ControlFlowTok{function}\NormalTok{(x) test2[,x]}\OperatorTok{/}\NormalTok{test2[,x}\DecValTok{-1}\NormalTok{])}
\NormalTok{    tmp =}\StringTok{ }\NormalTok{tmp[}\OperatorTok{-}\DecValTok{1}\NormalTok{]}
    \KeywordTok{names}\NormalTok{(sds) =}\StringTok{ }\NormalTok{tmp}

\NormalTok{  \}}\ControlFlowTok{else}\NormalTok{\{}
    \KeywordTok{stop}\NormalTok{(}\StringTok{"method need to be selected from 'reference','other','previous'"}\NormalTok{)}
\NormalTok{  \}}

  \ControlFlowTok{if}\NormalTok{(cutoff}\OperatorTok{<}\DecValTok{1}\NormalTok{)\{}
\NormalTok{    topdf =}\StringTok{ }\KeywordTok{nrow}\NormalTok{(df)}\OperatorTok{*}\NormalTok{cutoff}
\NormalTok{    sdtop =}\StringTok{ }\KeywordTok{lapply}\NormalTok{(tmp,}\ControlFlowTok{function}\NormalTok{(x) }\KeywordTok{names}\NormalTok{(sds[[x]][}\KeywordTok{order}\NormalTok{(sds[[x]],}\DataTypeTok{decreasing =}\NormalTok{ T)[}\DecValTok{1}\OperatorTok{:}\NormalTok{topdf]]))}
\NormalTok{  \}}\ControlFlowTok{else}\NormalTok{\{}
\NormalTok{    sdtop =}\StringTok{ }\KeywordTok{lapply}\NormalTok{(tmp,}\ControlFlowTok{function}\NormalTok{(x) }\KeywordTok{names}\NormalTok{(sds[[x]][sds[[x]]}\OperatorTok{>}\NormalTok{cutoff]))}
\NormalTok{  \}}

  \ControlFlowTok{if}\NormalTok{(method }\OperatorTok{==}\StringTok{ 'reference'}\NormalTok{) tmp =}\StringTok{ }\NormalTok{tmp[}\OperatorTok{-}\DecValTok{1}\NormalTok{]}
  \KeywordTok{names}\NormalTok{(sdtop) =}\StringTok{ }\NormalTok{tmp}
\NormalTok{  subdf =}\StringTok{ }\KeywordTok{lapply}\NormalTok{(tmp,}\ControlFlowTok{function}\NormalTok{(x) df[,}\KeywordTok{as.character}\NormalTok{(samplesL[[x]])])}
  \KeywordTok{names}\NormalTok{(subdf) =}\StringTok{ }\NormalTok{tmp}
\NormalTok{  subm =}\StringTok{ }\KeywordTok{lapply}\NormalTok{(}\KeywordTok{names}\NormalTok{(subdf), }\ControlFlowTok{function}\NormalTok{(x) }\KeywordTok{subset}\NormalTok{(subdf[[x]],}\KeywordTok{row.names}\NormalTok{(subdf[[x]]) }\OperatorTok\StringTok{ }\NormalTok{sdtop[[x]]))}
  \KeywordTok{names}\NormalTok{(subm)  =}\StringTok{ }\NormalTok{tmp}
  \KeywordTok{return}\NormalTok{(subm)}
\NormalTok{\}}
\end{Highlighting}
\end{Shaded}

get an igraph object based on Pearson Correlation Coefficiency(PCC)

@description get a igraph object based on Pearson Correlation
Coefficiency(PCC)

@param optimal a list of count matrix @param p a numeric cutoff @return
a list of an igraph object @export @examples igraphL \textless-
getNetwork(test, p=1)

@importFrom stringr psych igraph

@author Zhezhen Wang and Biniam Feleke

\begin{Shaded}
\begin{Highlighting}[]
\NormalTok{getNetwork <-}\StringTok{ }\ControlFlowTok{function}\NormalTok{(optimal, }\DataTypeTok{p =} \FloatTok{0.05}\NormalTok{)\{}
\NormalTok{  rL =}\StringTok{ }\KeywordTok{lapply}\NormalTok{(optimal,}\ControlFlowTok{function}\NormalTok{(x) }\KeywordTok{corr.test}\NormalTok{(}\KeywordTok{t}\NormalTok{(x),}\DataTypeTok{adjust =} \StringTok{'fdr'}\NormalTok{,}\DataTypeTok{ci=}\NormalTok{F)}\OperatorTok{$}\NormalTok{r)}
  \KeywordTok{names}\NormalTok{(rL) =}\StringTok{ }\KeywordTok{names}\NormalTok{(optimal)}
\NormalTok{  pL =}\StringTok{ }\KeywordTok{lapply}\NormalTok{(optimal,}\ControlFlowTok{function}\NormalTok{(x) }\KeywordTok{corr.test}\NormalTok{(}\KeywordTok{t}\NormalTok{(x),}\DataTypeTok{adjust =} \StringTok{'fdr'}\NormalTok{,}\DataTypeTok{ci=}\NormalTok{F)}\OperatorTok{$}\NormalTok{p)}
  \ControlFlowTok{if}\NormalTok{(}\KeywordTok{is.null}\NormalTok{(}\KeywordTok{names}\NormalTok{(rL))) }\KeywordTok{stop}\NormalTok{(}\StringTok{'give names to the input list'}\NormalTok{)}
\NormalTok{  igraphL =}\StringTok{ }\KeywordTok{list}\NormalTok{()}
  \ControlFlowTok{for}\NormalTok{(i }\ControlFlowTok{in} \KeywordTok{names}\NormalTok{(rL))\{}
\NormalTok{    test =}\StringTok{ }\NormalTok{rL[[i]]}
\NormalTok{    test.p =}\StringTok{ }\NormalTok{pL[[i]]}
\NormalTok{    test[}\KeywordTok{lower.tri}\NormalTok{(test,}\DataTypeTok{diag =}\NormalTok{ T)] =}\StringTok{ }\OtherTok{NA}
    \CommentTok{#test.p[lower.tri(test,diag = T)] = 1}
\NormalTok{    tmp =}\StringTok{ }\KeywordTok{lapply}\NormalTok{(}\DecValTok{1}\OperatorTok{:}\KeywordTok{nrow}\NormalTok{(test),}\ControlFlowTok{function}\NormalTok{(x) test[x,test.p[x,]}\OperatorTok{<}\NormalTok{p])}
\NormalTok{    tmp_name =}\StringTok{ }\KeywordTok{lapply}\NormalTok{(}\DecValTok{1}\OperatorTok{:}\KeywordTok{nrow}\NormalTok{(test),}\ControlFlowTok{function}\NormalTok{(x) }\KeywordTok{which}\NormalTok{(test.p[x,]}\OperatorTok{<}\NormalTok{p))}
\NormalTok{    idx =}\StringTok{ }\KeywordTok{which}\NormalTok{(}\KeywordTok{lengths}\NormalTok{(tmp_name)}\OperatorTok{==}\DecValTok{1}\NormalTok{)}
    \ControlFlowTok{for}\NormalTok{(j }\ControlFlowTok{in}\NormalTok{ idx)\{}
      \KeywordTok{names}\NormalTok{(tmp[[j]]) =}\StringTok{ }\KeywordTok{names}\NormalTok{(tmp_name[[j]])}
\NormalTok{    \}}
    \KeywordTok{names}\NormalTok{(tmp) =}\StringTok{ }\KeywordTok{row.names}\NormalTok{(test)}
\NormalTok{    edges =}\StringTok{ }\KeywordTok{stack}\NormalTok{(}\KeywordTok{do.call}\NormalTok{(c,tmp))}
\NormalTok{    edges =}\StringTok{ }\KeywordTok{subset}\NormalTok{(edges, }\OperatorTok{!}\KeywordTok{is.na}\NormalTok{(values))}
\NormalTok{    tmp2 =}\StringTok{ }\KeywordTok{subset}\NormalTok{(edges,}\KeywordTok{grepl}\NormalTok{(}\StringTok{'}\CharTok{\textbackslash{}\textbackslash{}}\StringTok{.[1-9,A-z]}\CharTok{\textbackslash{}\textbackslash{}}\StringTok{.'}\NormalTok{,ind))}
    \ControlFlowTok{if}\NormalTok{(}\KeywordTok{nrow}\NormalTok{(tmp2)}\OperatorTok{!=}\DecValTok{0}\NormalTok{)\{}
\NormalTok{      tmp2}\OperatorTok{$}\NormalTok{node1 =}\StringTok{ }\KeywordTok{paste0}\NormalTok{(}\KeywordTok{str_split_fixed}\NormalTok{(tmp2}\OperatorTok{$}\NormalTok{ind,}\StringTok{'}\CharTok{\textbackslash{}\textbackslash{}}\StringTok{.'}\NormalTok{,}\DecValTok{3}\NormalTok{)[,}\DecValTok{1}\NormalTok{],}\StringTok{'.'}\NormalTok{,}\KeywordTok{str_split_fixed}\NormalTok{(tmp2}\OperatorTok{$}\NormalTok{ind,}\StringTok{'}\CharTok{\textbackslash{}\textbackslash{}}\StringTok{.'}\NormalTok{,}\DecValTok{3}\NormalTok{)[,}\DecValTok{2}\NormalTok{])}
\NormalTok{      tmp2}\OperatorTok{$}\NormalTok{node2 =}\StringTok{ }\KeywordTok{str_split_fixed}\NormalTok{(tmp2}\OperatorTok{$}\NormalTok{ind,}\StringTok{'}\CharTok{\textbackslash{}\textbackslash{}}\StringTok{.'}\NormalTok{,}\DecValTok{3}\NormalTok{)[,}\DecValTok{3}\NormalTok{]}
\NormalTok{    \}}
\NormalTok{    edges =}\StringTok{ }\KeywordTok{subset}\NormalTok{(edges,}\OperatorTok{!}\KeywordTok{grepl}\NormalTok{(}\StringTok{'}\CharTok{\textbackslash{}\textbackslash{}}\StringTok{.[1-9,A-z]}\CharTok{\textbackslash{}\textbackslash{}}\StringTok{.'}\NormalTok{,ind))}
\NormalTok{    edges}\OperatorTok{$}\NormalTok{node1 =}\StringTok{ }\KeywordTok{str_split_fixed}\NormalTok{(edges}\OperatorTok{$}\NormalTok{ind,}\StringTok{'}\CharTok{\textbackslash{}\textbackslash{}}\StringTok{.'}\NormalTok{,}\DecValTok{2}\NormalTok{)[,}\DecValTok{1}\NormalTok{]}
\NormalTok{    edges}\OperatorTok{$}\NormalTok{node2 =}\StringTok{ }\KeywordTok{str_split_fixed}\NormalTok{(edges}\OperatorTok{$}\NormalTok{ind,}\StringTok{'}\CharTok{\textbackslash{}\textbackslash{}}\StringTok{.'}\NormalTok{,}\DecValTok{2}\NormalTok{)[,}\DecValTok{2}\NormalTok{]}
\NormalTok{    edges =}\StringTok{ }\KeywordTok{rbind}\NormalTok{(edges,tmp2)}
    \KeywordTok{dim}\NormalTok{(edges) }\CommentTok{#[1] 1270    4}
    \KeywordTok{dim}\NormalTok{(edges) }\CommentTok{#[1] 583   4}
\NormalTok{    edges =}\StringTok{ }\NormalTok{edges[,}\KeywordTok{c}\NormalTok{(}\StringTok{'node1'}\NormalTok{,}\StringTok{'node2'}\NormalTok{,}\StringTok{'values'}\NormalTok{)]}
\NormalTok{    edges}\OperatorTok{$}\NormalTok{weight =}\StringTok{ }\KeywordTok{abs}\NormalTok{(edges}\OperatorTok{$}\NormalTok{values) }\CommentTok{# added in 1/8/2019}
    \CommentTok{#colnames(edges) = c('node1','node2','weight') # added in 12/18/2018}

\NormalTok{    nodes =}\StringTok{ }\KeywordTok{data.frame}\NormalTok{(}\KeywordTok{unique}\NormalTok{(}\KeywordTok{c}\NormalTok{(edges}\OperatorTok{$}\NormalTok{node1,edges}\OperatorTok{$}\NormalTok{node2)))}
    \KeywordTok{print}\NormalTok{(}\KeywordTok{paste0}\NormalTok{(i,}\StringTok{':'}\NormalTok{,}\KeywordTok{nrow}\NormalTok{(nodes),}\StringTok{' nodes'}\NormalTok{)) }\CommentTok{#[1] 48    1}
\NormalTok{    routes_igraph <-}\StringTok{ }\KeywordTok{graph_from_data_frame}\NormalTok{(}\DataTypeTok{d =}\NormalTok{ edges, }\DataTypeTok{vertices =}\NormalTok{ nodes, }\DataTypeTok{directed =}\NormalTok{ F)}
\NormalTok{    igraphL[[i]] =}\StringTok{ }\NormalTok{routes_igraph}
\NormalTok{  \}}
  \KeywordTok{return}\NormalTok{(igraphL)}
\NormalTok{\}}
\end{Highlighting}
\end{Shaded}

getCluster

@param igraphL a list of count matrix or a list of igraph object @param
steps number of steps @return A list of objects @export @examples
getCluster(igraphL, steps = 4)

@author Zhezhen Wang

\begin{Shaded}
\begin{Highlighting}[]
\NormalTok{getCluster <-}\StringTok{ }\ControlFlowTok{function}\NormalTok{(igraphL, }\DataTypeTok{steps =} \DecValTok{4}\NormalTok{)\{}
  \CommentTok{#   if(is.na(step))\{}
  \CommentTok{#     step = sapply(igraphL,function(x) nrow(as_data_frame(x, what="vertices"))*2)}
  \CommentTok{#     print(step)}
  \CommentTok{#   \}else}
  \ControlFlowTok{if}\NormalTok{(}\KeywordTok{length}\NormalTok{(steps) }\OperatorTok{==}\StringTok{ }\DecValTok{1} \OperatorTok{&}\StringTok{ }\NormalTok{steps }\OperatorTok\StringTok{ }\DecValTok{1} \OperatorTok{==}\StringTok{ }\DecValTok{0}\NormalTok{)\{               }\CommentTok{##grepl("^[1-9]\{1,\}$", step) only works for 1 digit}
\NormalTok{    steps =}\StringTok{ }\KeywordTok{rep}\NormalTok{(steps,}\KeywordTok{length}\NormalTok{(igraphL))}
\NormalTok{  \}}\ControlFlowTok{else} \ControlFlowTok{if}\NormalTok{(}\KeywordTok{length}\NormalTok{(steps) }\OperatorTok{!=}\StringTok{ }\DecValTok{1} \OperatorTok{|}\StringTok{ }\KeywordTok{length}\NormalTok{(steps) }\OperatorTok{!=}\StringTok{ }\KeywordTok{length}\NormalTok{(igraphL))\{}
    \KeywordTok{stop}\NormalTok{(}\StringTok{'check step: must be postive integer(s) of length 1 or length of igraphL'}\NormalTok{)}
\NormalTok{  \}}
\NormalTok{  groups =}\StringTok{ }\KeywordTok{list}\NormalTok{()}
  \ControlFlowTok{for}\NormalTok{(i }\ControlFlowTok{in} \DecValTok{1}\OperatorTok{:}\KeywordTok{length}\NormalTok{(igraphL))\{}
    \ControlFlowTok{if}\NormalTok{(}\KeywordTok{nrow}\NormalTok{(}\KeywordTok{as_data_frame}\NormalTok{(igraphL[[i]])) }\OperatorTok{!=}\StringTok{ }\DecValTok{0}\NormalTok{)\{}
\NormalTok{      groups[[i]] =}\StringTok{ }\KeywordTok{cluster_walktrap}\NormalTok{(igraphL[[i]],}\DataTypeTok{weight =} \KeywordTok{abs}\NormalTok{(}\KeywordTok{E}\NormalTok{(igraphL[[i]])}\OperatorTok{$}\NormalTok{weight),}\DataTypeTok{steps =}\NormalTok{ steps[i])}
\NormalTok{    \}}\ControlFlowTok{else}\NormalTok{\{}
\NormalTok{      groups[[i]] =}\StringTok{ }\OtherTok{NA}
\NormalTok{    \}}
\NormalTok{  \}}
  \CommentTok{#groups = lapply(1:length(igraphL), function(x) cluster_walktrap(igraphL[[x]],weight = abs(E(igraphL[[x]])$weight),steps = steps[x])) # changed weight to abs(PCC) 12/18/2018}
  \KeywordTok{names}\NormalTok{(groups) =}\StringTok{ }\KeywordTok{names}\NormalTok{(igraphL)}
  \KeywordTok{return}\NormalTok{(groups)}
\NormalTok{\}}
\end{Highlighting}
\end{Shaded}

get clusters of nodes by clustering methods

@param igraphL a list of count matrix or a list of igraph object @param
method a character need to be select from `rw', `hcm', `km', `pam', or
`natrual'. Uses `rw' by default for `hcm', using complete. for `km'
using `euclidean'. @param cutoff a numeric value, default NULL @param
countsL a list of count matrix @return A list of vertex id vectors that
can be used in plot function in igraph package as `mark.groups'
parameter @export @examples cluster \textless-
getCluster\_methods(igraphL) head(cluster)

@author Zhezhen Wang and Biniam Feleke

\begin{Shaded}
\begin{Highlighting}[]
\NormalTok{getCluster_methods <-}\StringTok{ }\ControlFlowTok{function}\NormalTok{(igraphL, }\DataTypeTok{method =} \StringTok{'rw'}\NormalTok{, }\DataTypeTok{cutoff =} \OtherTok{NULL}\NormalTok{)\{}
  \ControlFlowTok{if}\NormalTok{(method }\OperatorTok{==}\StringTok{ 'rw'}\NormalTok{)\{}
    \ControlFlowTok{if}\NormalTok{(}\KeywordTok{all}\NormalTok{(}\KeywordTok{sapply}\NormalTok{(igraphL,class) }\OperatorTok{!=}\StringTok{ 'igraph'}\NormalTok{))}
      \KeywordTok{stop}\NormalTok{(}\StringTok{'random walk clustering needs a list of igraph object which can be obtained using getNetwork'}\NormalTok{)}
    \ControlFlowTok{if}\NormalTok{(}\KeywordTok{is.null}\NormalTok{(cutoff)) cutoff =}\StringTok{ }\DecValTok{4}
    \ControlFlowTok{if}\NormalTok{(cutoff}\OperatorTok\DecValTok{1} \OperatorTok{!=}\DecValTok{0}\NormalTok{) }\KeywordTok{warning}\NormalTok{(}\StringTok{'Please provide a integer as "cutoff" for the cluster method random walk'}\NormalTok{)}
\NormalTok{    groups =}\StringTok{ }\KeywordTok{getCluster}\NormalTok{(igraphL,cutoff)}
\NormalTok{  \}}\ControlFlowTok{else} \ControlFlowTok{if}\NormalTok{(method }\OperatorTok{==}\StringTok{ 'hcm'}\NormalTok{)\{}
    \ControlFlowTok{if}\NormalTok{(}\KeywordTok{all}\NormalTok{(}\OperatorTok{!}\KeywordTok{sapply}\NormalTok{(igraphL,class) }\OperatorTok\StringTok{ }\KeywordTok{c}\NormalTok{(}\StringTok{'matrix'}\NormalTok{,}\StringTok{'data.frame'}\NormalTok{)))}
      \KeywordTok{stop}\NormalTok{(}\StringTok{'random walk clustering needs a list of igraph object as the 1st argument which can be obtained using getNetwork'}\NormalTok{)}
\NormalTok{    testL =}\StringTok{ }\KeywordTok{lapply}\NormalTok{(igraphL, }\ControlFlowTok{function}\NormalTok{(x) }\KeywordTok{corr.test}\NormalTok{(}\KeywordTok{t}\NormalTok{(x),}\DataTypeTok{adjust =} \StringTok{'fdr'}\NormalTok{,}\DataTypeTok{ci=}\NormalTok{F)}\OperatorTok{$}\NormalTok{r)}
\NormalTok{    groups =}\StringTok{ }\KeywordTok{lapply}\NormalTok{(}\DecValTok{1}\OperatorTok{:}\KeywordTok{length}\NormalTok{(testL), }\ControlFlowTok{function}\NormalTok{(x) }\KeywordTok{hclust}\NormalTok{(}\KeywordTok{dist}\NormalTok{(testL[[x]]), }\DataTypeTok{method =} \StringTok{"complete"}\NormalTok{))}
\NormalTok{  \}}\ControlFlowTok{else} \ControlFlowTok{if}\NormalTok{(method }\OperatorTok\StringTok{ }\KeywordTok{c}\NormalTok{(}\StringTok{'km'}\NormalTok{,}\StringTok{'pam'}\NormalTok{))\{}
    \ControlFlowTok{if}\NormalTok{(}\KeywordTok{all}\NormalTok{(}\OperatorTok{!}\KeywordTok{sapply}\NormalTok{(igraphL,class) }\OperatorTok\StringTok{ }\KeywordTok{c}\NormalTok{(}\StringTok{'matrix'}\NormalTok{,}\StringTok{'data.frame'}\NormalTok{)))}
      \KeywordTok{stop}\NormalTok{(}\StringTok{'random walk clustering needs a list of igraph object as the 1st argument which can be obtained using getNetwork'}\NormalTok{)}
\NormalTok{    testL =}\StringTok{ }\KeywordTok{lapply}\NormalTok{(igraphL, }\ControlFlowTok{function}\NormalTok{(x) }\KeywordTok{corr.test}\NormalTok{(}\KeywordTok{t}\NormalTok{(x),}\DataTypeTok{adjust =} \StringTok{'fdr'}\NormalTok{,}\DataTypeTok{ci=}\NormalTok{F)}\OperatorTok{$}\NormalTok{r)}
\NormalTok{    groups =}\StringTok{ }\KeywordTok{lapply}\NormalTok{(}\DecValTok{1}\OperatorTok{:}\KeywordTok{length}\NormalTok{(testL), }\ControlFlowTok{function}\NormalTok{(x) }\KeywordTok{KMedoids}\NormalTok{(testL[[x]],}\DecValTok{3}\NormalTok{,}\DataTypeTok{distance =} \StringTok{'euclidean'}\NormalTok{))}
\NormalTok{  \}}\ControlFlowTok{else} \ControlFlowTok{if}\NormalTok{(method }\OperatorTok{==}\StringTok{ 'natrual'}\NormalTok{)\{}
    \ControlFlowTok{if}\NormalTok{(}\KeywordTok{all}\NormalTok{(}\KeywordTok{sapply}\NormalTok{(igraphL,class) }\OperatorTok{!=}\StringTok{ 'igraph'}\NormalTok{))}
      \KeywordTok{stop}\NormalTok{(}\StringTok{'random walk clustering needs a list of igraph object as the 1st argument which can be obtained using getNetwork'}\NormalTok{)}
\NormalTok{    groups =}\StringTok{ }\KeywordTok{lapply}\NormalTok{(}\DecValTok{1}\OperatorTok{:}\KeywordTok{length}\NormalTok{(igraphL), }\ControlFlowTok{function}\NormalTok{(x) }\KeywordTok{components}\NormalTok{(igraphL[[x]])}\OperatorTok{$}\NormalTok{membership)}

\NormalTok{  \}}\ControlFlowTok{else}\NormalTok{(}\KeywordTok{stop}\NormalTok{(}\StringTok{'please select from "rw", "hcm","km", "pam", "natrual" as method'}\NormalTok{))}
  \KeywordTok{return}\NormalTok{(groups)}
\NormalTok{\}}

\CommentTok{# step3: calulate CI}
\end{Highlighting}
\end{Shaded}

get CI score

@description get CI score

@param groups A list of vertex id vectors @param countsL a list of
numeric count matrix @param plot a boolean to decide if plot a bar plot
of CI scores or not. Default TRUE @param adjust.size a boolean to decide
if CI score should be adjust by size or not. Default FALSE @param ylim a
vector needed if the output barplots need to be on the same y scale
@param nr the number of rows to plot @param nc the number of column to
plot @param order the order of the barplot. default NULL which is using
the input list order @return a list of CI score and their components
@export @examples par(mfrow \textless- c(1,5)) membersL\_noweight
\textless- getCI(cluster,test,adjust.size = F,ylim = c(0,8))
head(membersL\_noweight) maxCIms \textless-
getMaxCImember(membersL\_noweight{[}{[}1{]}{]},membersL\_noweight{[}{[}2{]}{]},min
=3) head(maxCIms)

@importFrom psych igraph @author Zhezhen Wang

\begin{Shaded}
\begin{Highlighting}[]
\CommentTok{#}

\NormalTok{getCI <-}\StringTok{ }\ControlFlowTok{function}\NormalTok{(groups,countsL,}\DataTypeTok{plot =} \OtherTok{TRUE}\NormalTok{, }\DataTypeTok{adjust.size =} \OtherTok{FALSE}\NormalTok{,}\DataTypeTok{ylim =} \OtherTok{NULL}\NormalTok{,}\DataTypeTok{nr=}\DecValTok{1}\NormalTok{,}\DataTypeTok{nc =} \KeywordTok{length}\NormalTok{(countsL),}\DataTypeTok{order =} \OtherTok{NULL}\NormalTok{)\{}
  \ControlFlowTok{if}\NormalTok{(}\KeywordTok{any}\NormalTok{(}\KeywordTok{sapply}\NormalTok{(groups,class) }\OperatorTok{==}\StringTok{ "communities"}\NormalTok{))\{}
\NormalTok{    membersL =}\StringTok{ }\KeywordTok{lapply}\NormalTok{(groups,membership)\}}
  \ControlFlowTok{else}\NormalTok{\{}
\NormalTok{    membersL =}\StringTok{ }\NormalTok{groups}
\NormalTok{  \}}
\NormalTok{  CIl =}\StringTok{ }\NormalTok{PCCol =}\StringTok{ }\NormalTok{PCCl =}\StringTok{ }\NormalTok{sdl =}\KeywordTok{list}\NormalTok{()}
  \KeywordTok{names}\NormalTok{(membersL) =}\StringTok{ }\KeywordTok{names}\NormalTok{(countsL) }\CommentTok{# probably need to be changed to names(groups) instead of names(countsL)}
  \ControlFlowTok{if}\NormalTok{(}\KeywordTok{is.null}\NormalTok{(}\KeywordTok{names}\NormalTok{(countsL))) }\KeywordTok{warning}\NormalTok{(}\StringTok{'No names provided for the countsL'}\NormalTok{)}
  \ControlFlowTok{if}\NormalTok{(}\KeywordTok{is.null}\NormalTok{(nc)) nc =}\StringTok{ }\KeywordTok{length}\NormalTok{(groups)}
  \ControlFlowTok{if}\NormalTok{(plot) }\KeywordTok{par}\NormalTok{(}\DataTypeTok{mfrow =}\KeywordTok{c}\NormalTok{(nr,nc))}

  \ControlFlowTok{if}\NormalTok{(}\KeywordTok{is.null}\NormalTok{(order))\{}
\NormalTok{    loop =}\StringTok{ }\DecValTok{1}\OperatorTok{:}\KeywordTok{length}\NormalTok{(membersL)}
\NormalTok{  \}}\ControlFlowTok{else}\NormalTok{\{}
\NormalTok{    loop =}\StringTok{ }\NormalTok{order}
\NormalTok{  \}}

  \ControlFlowTok{for}\NormalTok{(i }\ControlFlowTok{in}\NormalTok{ loop)\{}
\NormalTok{    test =}\StringTok{ }\NormalTok{membersL[[i]]}
    \ControlFlowTok{if}\NormalTok{(}\KeywordTok{all}\NormalTok{(}\KeywordTok{is.na}\NormalTok{(test)))\{}
\NormalTok{      CI =}\StringTok{ }\NormalTok{sdL =}\StringTok{ }\NormalTok{PCC =}\StringTok{ }\NormalTok{PCCo =}\StringTok{ }\OtherTok{NA}
\NormalTok{    \}}
    \ControlFlowTok{else}\NormalTok{\{}
\NormalTok{      test.counts =}\StringTok{ }\NormalTok{countsL[[i]]}
\NormalTok{      m =}\StringTok{ }\KeywordTok{lapply}\NormalTok{(}\DecValTok{1}\OperatorTok{:}\KeywordTok{max}\NormalTok{(test),}\ControlFlowTok{function}\NormalTok{(x) }\KeywordTok{subset}\NormalTok{(test.counts, }\KeywordTok{row.names}\NormalTok{(test.counts) }\OperatorTok\StringTok{ }\KeywordTok{names}\NormalTok{(test[test}\OperatorTok{==}\NormalTok{x])))}
\NormalTok{      comple =}\StringTok{ }\KeywordTok{lapply}\NormalTok{(}\DecValTok{1}\OperatorTok{:}\KeywordTok{max}\NormalTok{(test),}\ControlFlowTok{function}\NormalTok{(x) }\KeywordTok{subset}\NormalTok{(test.counts, }\OperatorTok{!}\KeywordTok{row.names}\NormalTok{(test.counts) }\OperatorTok\StringTok{ }\KeywordTok{names}\NormalTok{(test[test}\OperatorTok{==}\NormalTok{x])))}
      \KeywordTok{names}\NormalTok{(m) =}\StringTok{ }\KeywordTok{names}\NormalTok{(comple) =}\StringTok{ }\NormalTok{letters[}\DecValTok{1}\OperatorTok{:}\KeywordTok{max}\NormalTok{(test)]}

      \CommentTok{#PCCo = mapply(function(x,y) abs(cor(t(x),t(y))),comple,m)}
\NormalTok{      PCCo =}\StringTok{ }\KeywordTok{lapply}\NormalTok{(}\KeywordTok{names}\NormalTok{(m), }\ControlFlowTok{function}\NormalTok{(x) }\KeywordTok{abs}\NormalTok{(}\KeywordTok{cor}\NormalTok{(}\KeywordTok{t}\NormalTok{(comple[[x]]),}\KeywordTok{t}\NormalTok{(m[[x]]))))}
\NormalTok{      PCCo_avg =}\StringTok{ }\KeywordTok{sapply}\NormalTok{(PCCo,mean)}

\NormalTok{      PCC =}\StringTok{ }\KeywordTok{lapply}\NormalTok{(m,}\ControlFlowTok{function}\NormalTok{(x) }\KeywordTok{abs}\NormalTok{(}\KeywordTok{cor}\NormalTok{(}\KeywordTok{t}\NormalTok{(x))))}
      \CommentTok{#for(j in 1:length(PCC))\{}
      \CommentTok{#  PCC[[j]][lower.tri(PCC[[j]],diag = T)] = NA}
      \CommentTok{#\}}
\NormalTok{      PCC_avg =}\StringTok{ }\KeywordTok{sapply}\NormalTok{(PCC,}\ControlFlowTok{function}\NormalTok{(x) (}\KeywordTok{sum}\NormalTok{(x,}\DataTypeTok{na.rm =}\NormalTok{ T)}\OperatorTok{-}\KeywordTok{nrow}\NormalTok{(x))}\OperatorTok{/}\NormalTok{(}\KeywordTok{nrow}\NormalTok{(x)}\OperatorTok{^}\DecValTok{2}\OperatorTok{-}\KeywordTok{nrow}\NormalTok{(x)))}
      \CommentTok{#PCC_avg = sapply(PCC,function(x) mean(x,na.rm = T))}
\NormalTok{      sdL =}\StringTok{ }\KeywordTok{lapply}\NormalTok{(m, }\ControlFlowTok{function}\NormalTok{(x) }\KeywordTok{apply}\NormalTok{(x,}\DecValTok{1}\NormalTok{,sd))}
      \ControlFlowTok{if}\NormalTok{(adjust.size)\{}
\NormalTok{        CI =}\StringTok{ }\KeywordTok{mapply}\NormalTok{(}\ControlFlowTok{function}\NormalTok{(x,y,z,w) }\KeywordTok{mean}\NormalTok{(x)}\OperatorTok{*}\NormalTok{(y}\OperatorTok{/}\NormalTok{z)}\OperatorTok{*}\KeywordTok{sqrt}\NormalTok{(}\KeywordTok{nrow}\NormalTok{(w)), sdL,PCC_avg,PCCo_avg,m)}
\NormalTok{      \}}\ControlFlowTok{else}\NormalTok{\{}
\NormalTok{        CI =}\StringTok{ }\KeywordTok{mapply}\NormalTok{(}\ControlFlowTok{function}\NormalTok{(x,y,z) }\KeywordTok{mean}\NormalTok{(x)}\OperatorTok{*}\NormalTok{(y}\OperatorTok{/}\NormalTok{z), sdL,PCC_avg,PCCo_avg)}
\NormalTok{      \}}

      \ControlFlowTok{if}\NormalTok{(plot) \{}
\NormalTok{        tn =}\StringTok{ }\KeywordTok{ifelse}\NormalTok{(}\KeywordTok{is.null}\NormalTok{(order),}\KeywordTok{names}\NormalTok{(membersL)[i],i)}
\NormalTok{        cex =}\StringTok{ }\KeywordTok{ifelse}\NormalTok{(}\KeywordTok{length}\NormalTok{(CI)}\OperatorTok{>}\DecValTok{20}\NormalTok{,}\FloatTok{0.7}\NormalTok{,}\DecValTok{1}\NormalTok{) }\CommentTok{## added 1/9/2019}
        \CommentTok{## 3/1/2019 changed legend 'n=' to #letters=}
        \KeywordTok{barplot}\NormalTok{(CI,}\DataTypeTok{legend =} \KeywordTok{paste0}\NormalTok{(}\StringTok{'#'}\NormalTok{,}\KeywordTok{names}\NormalTok{(m),}\StringTok{'='}\NormalTok{,}\KeywordTok{sapply}\NormalTok{(m,nrow)),}\DataTypeTok{col =} \KeywordTok{rainbow}\NormalTok{(}\KeywordTok{length}\NormalTok{(m), }\DataTypeTok{alpha =} \FloatTok{0.3}\NormalTok{),}
                \DataTypeTok{main =}\NormalTok{ tn,}\DataTypeTok{ylab =} \StringTok{'CI'}\NormalTok{,}\DataTypeTok{xlab=}\StringTok{'modules'}\NormalTok{,}\DataTypeTok{args.legend =} \KeywordTok{list}\NormalTok{(}\DataTypeTok{cex =}\NormalTok{ cex),}\DataTypeTok{ylim =}\NormalTok{ylim)}
\NormalTok{      \}}
\NormalTok{    \}}
\NormalTok{    CIl[[i]] =}\StringTok{ }\NormalTok{CI}
\NormalTok{    sdl[[i]] =}\StringTok{ }\NormalTok{sdL}
\NormalTok{    PCCl[[i]] =}\StringTok{ }\NormalTok{PCC}
\NormalTok{    PCCol[[i]] =}\StringTok{ }\NormalTok{PCCo}
\NormalTok{  \}}
  \KeywordTok{names}\NormalTok{(CIl) =}\StringTok{ }\KeywordTok{names}\NormalTok{(PCCol) =}\StringTok{ }\KeywordTok{names}\NormalTok{(PCCl) =}\StringTok{ }\KeywordTok{names}\NormalTok{(sdl) =}\StringTok{ }\KeywordTok{names}\NormalTok{(countsL)}
  \KeywordTok{return}\NormalTok{(}\KeywordTok{list}\NormalTok{(}\DataTypeTok{members =}\NormalTok{ membersL,}\DataTypeTok{CI =}\NormalTok{ CIl,}\DataTypeTok{sd =}\NormalTok{ sdl,}\DataTypeTok{PCC=}\NormalTok{PCCl,}\DataTypeTok{PCCo=}\NormalTok{PCCol))}
\NormalTok{\}}
\end{Highlighting}
\end{Shaded}

get the index and members with the maximum CI score

@description get the index and members with the maximum CI score

@param membersL a list of characters, the first element of output from
function getCI @param CIl alist of numerics, the secned element of
output from function getCI @param minsize a numeric value of minimum
size allowed for a cluster @return a list of index and members @export
@examples membersL\_noweight \textless- getCI(cluster,test,adjust.size =
F,ylim = c(0,8)) maxCIms \textless-
getMaxCImember(membersL\_noweight{[}{[}1{]}{]},membersL\_noweight{[}{[}2{]}{]},min
=3) head(maxCIms)

@author Zhezhen Wang

\begin{Shaded}
\begin{Highlighting}[]
\NormalTok{getMaxCImember <-}\StringTok{ }\ControlFlowTok{function}\NormalTok{(membersL,CIl,}\DataTypeTok{minsize =} \DecValTok{1}\NormalTok{)\{}
\NormalTok{  listn =}\StringTok{ }\KeywordTok{names}\NormalTok{(membersL)}
  \ControlFlowTok{if}\NormalTok{(}\OperatorTok{!}\NormalTok{minsize }\OperatorTok{<}\DecValTok{1}\NormalTok{)\{}
\NormalTok{    minsize =}\StringTok{ }\NormalTok{minsize}\DecValTok{-1}
\NormalTok{    CIl =}\StringTok{ }\KeywordTok{lapply}\NormalTok{(}\DecValTok{1}\OperatorTok{:}\KeywordTok{length}\NormalTok{(membersL),}\ControlFlowTok{function}\NormalTok{(x) }\KeywordTok{ifelse}\NormalTok{(}\KeywordTok{table}\NormalTok{(membersL[[x]])}\OperatorTok{>}\NormalTok{minsize,CIl[[x]],}\OtherTok{NA}\NormalTok{))}
    \CommentTok{# bug unsolved case     1     2     3     4     5     6}
    \CommentTok{#                      TRUE  TRUE  TRUE  FALSE  TRUE FALSE}
    \CommentTok{#CIl = lapply(1:length(membersL),function(x) CIl[[x]][table(membersL[[x]])>minsize])}
\NormalTok{    module_keep =}\StringTok{ }\KeywordTok{lapply}\NormalTok{(}\DecValTok{1}\OperatorTok{:}\KeywordTok{length}\NormalTok{(membersL), }\ControlFlowTok{function}\NormalTok{(x) }\KeywordTok{names}\NormalTok{(}\KeywordTok{table}\NormalTok{(membersL[[x]])[}\KeywordTok{table}\NormalTok{(membersL[[x]])}\OperatorTok{>}\NormalTok{(minsize}\DecValTok{-1}\NormalTok{)]))}
\NormalTok{    membersL =}\StringTok{ }\KeywordTok{lapply}\NormalTok{(}\DecValTok{1}\OperatorTok{:}\KeywordTok{length}\NormalTok{(membersL),}\ControlFlowTok{function}\NormalTok{(x) membersL[[x]][membersL[[x]] }\OperatorTok\StringTok{ }\NormalTok{module_keep[[x]]])}
\NormalTok{  \}}\ControlFlowTok{else}\NormalTok{(}\KeywordTok{stop}\NormalTok{(}\StringTok{'please provide a minimum size for the cluster, which should be integer that is larger than 0'}\NormalTok{))}

\NormalTok{  idx =}\StringTok{ }\KeywordTok{sapply}\NormalTok{(CIl,which.max)}
\NormalTok{  maxCI =}\StringTok{ }\KeywordTok{sapply}\NormalTok{(}\DecValTok{1}\OperatorTok{:}\KeywordTok{length}\NormalTok{(idx),}\ControlFlowTok{function}\NormalTok{(x) }\KeywordTok{names}\NormalTok{(membersL[[x]][membersL[[x]] }\OperatorTok{==}\StringTok{ }\NormalTok{idx[x]]))}
  \CommentTok{#lapply(names(memberL_weight[[1]]),function(x) names(memberL_weight[[1]][[x]][memberL_weight[[1]][[x]]==maxCIms[[1]][[x]]]))}
  \KeywordTok{names}\NormalTok{(maxCI) =}\StringTok{ }\NormalTok{listn}
  \KeywordTok{names}\NormalTok{(idx) =}\StringTok{ }\NormalTok{listn}
  \KeywordTok{return}\NormalTok{(}\KeywordTok{list}\NormalTok{(}\DataTypeTok{idx =}\NormalTok{ idx,}\DataTypeTok{members =}\NormalTok{ maxCI))}
\NormalTok{\}}
\end{Highlighting}
\end{Shaded}


\end{document}
